\documentclass[11pt]{article}
\usepackage{geometry}                % See geometry.pdf to learn the layout options. There are lots.
\geometry{letterpaper}                   % ... or a4paper or a5paper or ... 
%\geometry{landscape}                % Activate for for rotated page geometry
%\usepackage[parfill]{parskip}    % Activate to begin paragraphs with an empty line rather than an indent
\usepackage{graphicx}
\usepackage{amssymb}
\usepackage{amsmath}
\usepackage{epstopdf}
\usepackage{hyperref}
\DeclareGraphicsRule{.tif}{png}{.png}{`convert #1 `dirname #1`/`basename #1 .tif`.png}



\graphicspath{
{/Users/Andy/Cruises_Research/Analysis/Andy_Pickering/eq14_patch_gamma/figures/}
}

\title{Patch/Gamma Analysis for EQ14 chameleon patches}
\author{Andy Pickering}
%\date{}                                           % Activate to display a given date or no date



\begin{document}
\maketitle

\tableofcontents
\newpage

%~~~~~~~~~~~~~~~~~~~~~
\section{Overview}

The goal of this analysis is to compute mixing efficency ($\Gamma$) for patches in EQ14 chameleon profiles, and see if we obtain values close to $\Gamma=0.2$.

%~~~~~~~~~~~~~~~~~~~~~
\section{Data}

Data are made by the `Chameleon' microstructure profiler near the equator during the `EQ14' experiment. 
\medskip

I'm using the raw Chameleon data files in: \newline
\verb++

\medskip

All my analysis is in the main folder: \newline  \verb+//Users/Andy/Cruises_Research/Analysis/Andy_Pickering/eq14_patch_gamma/+


%~~~~~~~~~~~~~~~~~~~~~
\section{Methods}

\begin{itemize}

\item \verb+Process_??_rawprofiles_AP.m+  Processes raw Chameleon files and saves `cal2' files which have the raw/ high-res profiles of temp and salinity. These are used to identify patches. $\chi$ and $\epsilon$ are not computed for these.

\item \verb+FindPatches_eq14_Raw.m+ Identifies patches in the profiles made by \verb+Process_tiwe_rawprofiles_AP.m+, using potential temperature.

\item \verb+Compute_N2_dTdz_patches_eq14.m+ Computes $N^2$ and $T_z$ for patches, using several different methods. SAves results in a structure `patches'.

\item \verb+run_eq14_forPatches.m+ Runs the Chameleon processing (including $\chi$ and $\epsilon$) for just the patches identified in \verb+FindPatches_tiwe_Raw.m+ . This calls \verb+average_data_PATCH_AP.m+ instead of \verb+average_data_gen1.m+.

%\item \verb+Run_tiwe_AP.m+ Runs the standard Chameleon processing, producing 1m avg quantities. I modifed this from \verb+run_tw91.m+.

\item \verb+combine_patch_profiles_eq14.m+ Combines the avg profiles made in \verb+??.m+ into a single structure with common depths.


\end{itemize}

\medskip

%%~~~~~~~
%\subsection{Overturns}
%Overturns (patches) are detected for each profile, using potential temperature.

%~~~~~~~
\subsection{dTdz}

Temperature gradient is computed for each patch using the following methods:
\begin{enumerate}
\item $dtdz_{range}$ : Take the range of T over the patch and divided by patch height
\item $dtdz_{line}$ : Fit a straight line to sorted T using \verb+polyfit+
\item $dtdz_{bulk}$ : Use the 'bulk gradient' from Smyth et al 2001, which is the rms fluctuation from the background (sorted) temperature, divided by the thorpe scale (the rms re-ordering distances).
\end{enumerate}


%~~~~~~~
\subsection{N2}

$N^2$ is computed for each patch using the following methods:
\begin{enumerate}
\item $N^2_{range}$ : Take the range of potential density over the patch divided by the patch height ($d\rho/dz$), then compute $N^2=\frac{-g}{\rho_o}\frac{d\rho}{dz}$ where $\rho_o$ is the mean potential density over the patch.
\item $N^2_{line}$ : Fit a straight line to sorted potential density using polyfit to get $d\rho/dz$, then compute N2.
\item $N^2_{bulk}$ : Use 'bulk gradient' . This is calculated from the bulk $T_z$, using a linear fit between density and temperature.
\item $N^2_4$ : Compute $N^2$ from the sorted profile (sorted by potential density) using \verb+sw_bfreq+, then take average over the patch. I believe this method is used by some commonly-used overturn codes.
\end{enumerate}


%~~~~~~~
\subsection{Mixing Efficiency}

Mixing Efficiency $\Gamma$ is computed from the following equation using differerent $N^2$ and $dT/dz$ values.
\begin{equation}
\Gamma=\frac{N^2 \chi}{2\epsilon T_{z}^{2}} 
\end{equation}
$\chi$ and $\epsilon$ are computed over each patch from the Chameleon data. Gamma is computed for the following 4 combinations:
\begin{enumerate}
\item  $\Gamma_{range}$ : N^{2}_{range}$, $dtdz_{range}$
\item  $\Gamma_{line}$ : N^{2}_{line}$, $dtdz_{line}$
\item  $\Gamma_{bulk}$ : N^{2}_{bulk}$, $dtdz_{bulk}$
\item  $\Gamma_{range}$ : N^{2}_{4}$, $dtdz_{line}$
\end{enumerate}
Values where $\epsilon$ is below the noise floor of $log_{10}[\epsilon]=-8.5$ are discarded.



%~~~~~~~~~~~~~~~~~~~~~
\section{Results}

\begin{itemize}
%\item For some reason many $\chi$ values below 150db are bad/missing? Not sure why.
%\item The median $\Gamma$ computed using the 1m avg data is $0.063$ (Figure \ref{avggam})}.
%\item Gamma computed over patches w/ linear fits is slightly higher than the binned gamma, but still less than $0.2$ (Figure \ref{patchgam}).
\end{itemize}
%

%\begin{figure}[htbp]
%\includegraphics[scale=0.8]{tiwe_avgCombine_N2_dtdz_chi_eps.png}
%\caption{Pcolor of the combined 1m avg chameleon data for TIWE. * Note for some reason many $\chi$ values below 150db are bad/missing.}
%\label{}
%\end{figure}
%
%\begin{figure}[htbp]
%\includegraphics[scale=0.8]{tiwe_avgCombine_gamma.png}
%\caption{Histogram of $\Gamma$ for 1m avg chameleon profiles. Vertical dashed line shows $\Gamma=0.2$.}
%\label{avggam}
%\end{figure}
%%
%
%
%\begin{figure}[htbp]
%\includegraphics[scale=0.8]{tiwe_minOT_25_usetemp_1_gammas_hist_yday_307_329.png}
%\caption{Histogram of $\Gamma$ for patches, using different estimates of $N^2$ and $T_z$. Vertical dashed line shows $\Gamma=0.2$. For all profiles.}
%\label{patchgam}
%\end{figure}
%
%
%\begin{figure}[htbp]
%\includegraphics[scale=0.8]{tiwe_minOT_25_usetemp_1_gammas_hist_yday_324_327.png}
%\caption{Histogram of $\Gamma$ for patches, using different estimates of $N^2$ and $T_z$. Vertical dashed line shows $\Gamma=0.2$. For profiles on yday 324-327.}
%\label{patchgam}
%\end{figure}




\clearpage
%~~~~~~~~~~~~~~~~
\subsection{Variation of $\Gamma$ over time}

To investigate whether $\Gamma$ varies over time, I plotted $\Gamma$ vs yday (Figure \ref{gamvscnum}).% It looks like the median $\Gamma$ is smaller than 0.2 for ydays less than 315, and then about equal to 0.2 after that (a few days are abnormal and might not have many profiles).

\begin{figure}[htbp]
\includegraphics[scale=0.8]{eq14_minOT_25_usetemp_1_gam_vs_cnum.png}
\caption{Plot of $\Gamma$ for patches vs cast number. Vertical line is $\Gamma=0.2$. Red circles are the median value for each cast.}
\label{gamvscnum}
\end{figure}



\end{document}  


